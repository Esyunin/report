%!TEX root = ../diplom.tex
\newgeometry{left=30mm,right=15mm,top=15mm,bottom=25mm,bindingoffset=0cm,headheight=15pt}
\begin{titlepage}
%\begin{spacing}{1}
	\fontsize{13pt}{13pt} \selectfont
	{\centering
	\linespread{1}
	\noindent{\fontsize{11pt}{11pt}\textbf{ МИНИСТЕРСТВО НАУКИ И ВЫСШЕГО ОБРАЗОВАНИЯ РОССИЙСКОЙ} \\[0.2em]  \textbf{ ФЕДЕРАЦИИ}}\\[13pt]
%	\begin{spacing}{1.5}
		{\fontsize{13pt}{13pt} \selectfont\bf  Федеральное государственное автономное образовательное учреждение \\
		высшего образования \\
		<<Национальный исследовательский \\ Нижегородский
		государственный университет им. Н.И. Лобачевского>>
		}\\[12pt] 
%	\end{spacing}
	{\fontsize{13pt}{13pt} \selectfont
	Радиофизический факультет\\
	Кафедра теории колебаний и автоматического регулирования\\[28pt]
	Направление <<Радиофизика>>\\
	\vspace{30pt}
	ОТЧЕТ ПО УЧЕБНОЙ ПРАКТИКЕ\\}
	\vspace{15pt}
	{\fontsize{15pt}{15pt}
	Практика по получению первичных профессиональных\\
	умений и навыков
	\vspace{15pt}
	}\\
	{\fontsize{13pt}{13pt}\textbf{
		ИССЛЕДОВАНИЕ ДИНАМИКИ НЕЙРОНА ПОД ВОЗДЕЙСТВИЕМ ШУМА
	}}\\
	\vspace{48pt}}\fontsize{12pt}{12pt} \selectfont
	\noindent
	Научный руководитель:\\[0.4em]
	доцент, к.ф.-м.н.,\hfill \rule{2cm}{1pt} Клиньшов В.В. \hphantom{,}\\[40pt]
	%
	%
	Студент 1-го курса магистратуры:
	\hfill \rule{2cm}{1pt} Есюнин Д.В. \hphantom{aa,}\\[30pt]
	\vfill
	\centering
	Нижний Новгород\\[0.4em]
	2020 год
% \noindent\tikz[remember picture,overlay] {
% % % \draw[draw=none,fill=black!50!gray]  rectangle ;
% % \draw [step=1.0cm,gray] (current page.north east) grid (current page.south west);
% \node[opacity=1,inner sep=0pt] at (current page.center){\includegraphics[width=\paperwidth,height=\paperheight]{fig/apt.png}
% };}
%\end{spacing}
\end{titlepage}
\clearpage
\restoregeometry
